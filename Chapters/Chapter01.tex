\chapter{はじめに 序論}



論文の組版において \LaTeX やWordと格闘しておられる方も多いのではないでしょうか。
本テンプレートは、効率的に論文を執筆できるように設計されたものです。\texttt{classicthesis} パッケージ バージョン 4.8 (\url{https://ctan.org/pkg/classicthesis?lang=en}) をベースに、我々の執筆ニーズに合わせてカスタマイズしたものです。

主な変更点は以下の通りです:
\begin{itemize}
    \item APA 第7版の引用スタイル
    \item 日本語著者名のカスタムフォーマット(複数の著者の間に「・」を使用)
    \item 日本語文献のソート機能の強化
    \item 多言語サポートの向上
\end{itemize}

本章では、テンプレートの機能と基本的な使用方法の概要を説明します。\autoref{ch:examples} では、実用的な例を用いて様々な \LaTeX の機能を紹介し、\autoref{ch:settings} では、テンプレートの設定オプションについて詳しく説明します。

\section{クイックスタート (要約)}

\textit{すぐに執筆を始めたいですか?} 基本的な手順は以下の通りです:

\begin{enumerate}
    \item \textit{情報の入力}: \texttt{classicthesis-config.tex} (54行目以降) を編集し、名前、タイトル、所属などを入力します。

    \item \textit{執筆}: \texttt{Chapters/Chapter01.tex}、\texttt{Chapter02.tex} などに章ごとのコンテンツを記述します。

    \item \textit{文献の構築}: \texttt{Bibliography.bib} (または \texttt{part1.bib}、\texttt{part2.bib} など) に参考文献エントリを追加します。

    \item \textit{画像の追加}: 図表を \texttt{gfx/} フォルダに配置します。

    \item \textit{コンパイル}: メニュー (左上) を開き、コンパイラを XeLaTeX に設定します。

\end{enumerate}

以上です!詳細な説明については、以下を読み進めてください。


\section{日本語対応}

本テンプレートでは、日本語入力をより良くサポートするために XeLaTeX エンジンを使用しています。オリジナルの ClassicThesis テンプレートは pdfLaTeX で最適に動作しますが、XeLaTeX は特に CJK (中国語、日本語、韓国語) 言語において優れた多言語機能を提供します。

論文が主に英語で、日本語のコンテンツが最小限の場合は、プリアンブルに以下を追加することで pdfLaTeX を使用することも可能です:

\begin{lstlisting}
%*************************************
\usepackage{CJKutf8}
%*************************************
\end{lstlisting}

その後、本文中で日本語テキストを以下のように囲みます:

\begin{lstlisting}
%*************************************
\begin{CJK}{UTF8}{ipxm}
    Put your Japanese text here.
\end{CJK}
%*************************************
\end{lstlisting}

\LaTeX における日本語サポートの詳細については、\url{https://www.overleaf.com/learn/latex/Japanese} を参照してください。

\section{引用スタイル}

\subsection{英語文献}

標準的な BibTeX フォーマットに従って参考文献エントリを作成してください。以下は一般的なエントリタイプの例です:

\begin{lstlisting}

@incollection{fiorella2022,
  author    = {Fiorella, Logan and Mayer, Richard E.},
  title     = {The Generative Activity Principle in Multimedia Learning},
  booktitle = {The Cambridge Handbook of Multimedia Learning},
  editor    = {Mayer, Richard E. and Fiorella, Logan},
  edition   = {3},
  publisher = {Cambridge University Press},
  address   = {Cambridge},
  year      = {2022},
  pages     = {339--350},
  doi       = {10.1017/9781108894333.036}
}

@article{lusato2025,
  author    = {Lu, Jialiang and Sato, Reiko},
  title     = {Linguistic dimensions of comprehensibility and perceived fluency in {L2} speech across tasks of varying complexity},
  journal   = {Journal of Second Language Pronunciation},
  volume    = {11},
  number    = {2},
  year      = {2025},
  pages     = {240--266},
  doi       = {10.1075/jslp.24057.lu}
}

@book{mayer2021,
  author    = {Mayer, Richard E.},
  title     = {Multimedia Learning},
  edition   = {3},
  publisher = {Cambridge University Press},
  address   = {Cambridge},
  year      = {2020},
  doi       = {10.1017/9781316941355}
}

@INPROCEEDINGS{Miede2011,
    author = {Andr{\'e} Miede and G\"{o}khan \c{S}im\c{s}ek and Stefan Schulte
    and Abawi, Daniel F. and Julian Eckert and Ralf Steinmetz},
    title = {{R}evealing {B}usiness {R}elationships -- {E}avesdropping {C}ross-organizational
    {C}ollaboration in the {I}nternet of {S}ervices},
    booktitle = {Proceedings of the Tenth International Conference Wirtschaftsinformatik
    (WI 2011)},
    year = {2011},
    volume = {2},
    pages = {1083--1092},
    isbn = {978-1-4467-9236-0}
}
\end{lstlisting}


\subsection{日本語の文献}


\begin{lstlisting}
@article{tanaka2025,
  title        = {ダミーとミダーの関連性についての検討},
  author       = {田中, 太郎 and 山田, 花子},
  journal      = {日本科学会誌},
  volume       = {10},
  number       = {1},
  pages        = {10-25},
  year         = {2025},
  doi          = {10.1000/dummy.doi.123},
  langid       = {japanese},
  yomi         = {tanaka, taro and yamada, hanako}
}

@book{suzuki2024,
  title        = {バナナの基礎と応用},
  author       = {鈴木, 一郎},
  publisher    = {東工大出版会},
  address      = {東京},
  year         = {2024},
  langid       = {japanese},
  yomi         = {suzuki, ichiro}
}

@inproceedings{sato2023,
  title        = {東工大パワー丼における水菜と豚肉の配分},
  author       = {加藤, 次郎},
  booktitle    = {東工大学食学会第50回全国大会講演論文集},
  pages        = {100-102},
  year         = {2023},
  month        = aug,
  langid       = {japanese},
  yomi         = {kato, jiro}
}
\end{lstlisting}

\texttt{yomi} フィールドの形式によって、参考文献のソート順が決まります:

\begin{description}
    \item[ひらがな:] 日本語文献は英語文献の後に分けてリストされ、五十音順にソートされます。
    \item[ローマ字:] 日本語文献は英語文献と混在し、アルファベット順にソートされます。
\end{description}


\textit{注:} 中国語の文献でも \texttt{langid=japanese} を使用できますが、参考文献リストでは日本語フォントで表示されます。個人的にはこれで問題ないと考えています。
